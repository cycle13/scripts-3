%                **** IMPORTANT NOTICE *****
% This LaTeX file has been automatically produced by ProTeX v. 1.1
% Any changes made to this file will likely be lost next time
% this file is regenerated from its source. Send questions 
% to Arlindo da Silva, dasilva@gsfc.nasa.gov
 
%------------------------ PREAMBLE --------------------------
\documentclass[11pt]{article}
\usepackage{amsmath}
\usepackage{epsfig}
\usepackage{hangcaption}
\textheight     9in
\topmargin      0pt
\headsep        1cm
\headheight     0pt
\textwidth      6in
\oddsidemargin  0in
\evensidemargin 0in
\marginparpush  0pt
\pagestyle{myheadings}
\markboth{}{}
%-------------------------------------------------------------
\setlength{\parskip}{0pt}
\setlength{\parindent}{0pt}
\setlength{\baselineskip}{11pt}
 
%--------------------- SHORT-HAND MACROS ----------------------
\def\bv{\begin{verbatim}}
\def\ev{\end{verbatim}}
\def\be{\begin{equation}}
\def\ee{\end{equation}}
\def\bea{\begin{eqnarray}}
\def\eea{\end{eqnarray}}
\def\bi{\begin{itemize}}
\def\ei{\end{itemize}}
\def\bn{\begin{enumerate}}
\def\en{\end{enumerate}}
\def\bd{\begin{description}}
\def\ed{\end{description}}
\def\({\left (}
\def\){\right )}
\def\[{\left [}
\def\]{\right ]}
\def\<{\left  \langle}
\def\>{\right \rangle}
\def\cI{{\cal I}}
\def\diag{\mathop{\rm diag}}
\def\tr{\mathop{\rm tr}}
%-------------------------------------------------------------

\markboth{Left}{Source File: template\_introduction.txt,  Date: Tue May 13 06:41:06 CST 2014
}

 
 
%/////////////////////////////////////////////////////////////
\newpage

\markboth{Left}{Source File: template\_includefile.h,  Date: Tue May 13 06:41:06 CST 2014
}

 
%/////////////////////////////////////////////////////////////
\title{Front page template}
\author{{\sc Bob Yantosca and Philippe Le Sager}\\ {\em School of Epngineering and Applied Sciences, Harvard University}}
\date{May 23, 2008}
\begin{document}
\maketitle
\tableofcontents
\newpage
\section{Routine/Function Prologues} \label{app:ProLogues}

  \subsection{Fortran:  Module Interface GC\_SomethingIncludeFile.h }


  This include file contains the various parameters that will 
     allow the module and routine to do stuff to various things in various
     routines in various places.
  \\
  \\{\bf PUBLIC TYPES:}
\begin{verbatim}   TYPE t_GeosChemSomething
      !%%% declare stuff here %%%
   END TYPE t_GeosChemSomething\end{verbatim}{\bf PUBLIC MEMBER FUNCTIONS:}
\begin{verbatim}   None\end{verbatim}{\bf PUBLIC DATA MEMBERS:}
\begin{verbatim}   INTEGER(ESMF_KIND_I8), PUBLIC, PARAMETER :: myIntParam   ! INTEGER value
   REAL(ESMF_KIND_I8),    PUBLIC, PARAMETER :: myRealParam  ! REAL*8 value\end{verbatim}{\bf REVISION HISTORY:}
\begin{verbatim}    21 May 2008 - R. Yantosca - Initial Version\end{verbatim}{\bf REMARKS:}
\begin{verbatim} \end{verbatim}

\markboth{Left}{Source File: template\_module.F90,  Date: Tue May 13 06:41:06 CST 2014
}

 
%/////////////////////////////////////////////////////////////
 
\mbox{}\hrulefill\ 
 
 \subsection{Fortran:  Module Interface GC\_SomethingMod.F90 }


  This module contains the data type to declare a Something 
     object and the methods to work with the Something object. 
  \\
  \\{\bf INTERFACE:}
\begin{verbatim} MODULE GC_SomethingMod
   \end{verbatim}{\bf USES:}
\begin{verbatim}   USE ESMF_Mod
   IMPLICIT NONE\end{verbatim}{\bf PUBLIC TYPES:}
\begin{verbatim}   TYPE t_GeosChemSomething 
      !... declare stuff here
   END TYPE t_GeosChemSomething\end{verbatim}{\bf PUBLIC MEMBER FUNCTIONS:}
\begin{verbatim}   PUBLIC :: GC_SomethingRoutine1
   PUBLIC :: GC_SomethingFunction1\end{verbatim}{\bf PUBLIC DATA MEMBERS:}
\begin{verbatim}   INTEGER(ESMF_KIND_I4), PUBLIC :: myPublicVariable  ! public data variable\end{verbatim}{\bf REVISION HISTORY:}
\begin{verbatim}    21 May 2008 - R. Yantosca - Initial Version\end{verbatim}{\bf REMARKS:}
\begin{verbatim}   Protex is great!\end{verbatim}
 
%/////////////////////////////////////////////////////////////
 
\mbox{}\hrulefill\ 
 

  \subsubsection [GC\_SomethingRoutine1] {GC\_SomethingRoutine1}


  This routine does something to the input variable and returns 
   the result in the output variable.
  \\
  \\{\bf INTERFACE:}
\begin{verbatim}   SUBROUTINE GC_SomethingRoutine1( input, inpout, output, status )\end{verbatim}{\bf INPUT PARAMETERS:}
\begin{verbatim}     INTEGER(ESMF_KIND_I4), INTENT(IN) :: input   ! Input variable \end{verbatim}{\bf INPUT/OUTPUT PARAMETERS:}
\begin{verbatim}     INTEGER(ESMF_KIND_I4), INTENT(IN) :: inpout  ! In/out variable\end{verbatim}{\bf OUTPUT PARAMETERS:}
\begin{verbatim}     INTEGER(ESMF_KIND_I4), INTENT(IN) :: output  ! Output variable\end{verbatim}{\bf REVISION HISTORY:}
\begin{verbatim}    21 May 2008 - R. Yantosca - Initial Version\end{verbatim}{\bf REMARKS:}
\begin{verbatim}   Protex is great!\end{verbatim}
 
%/////////////////////////////////////////////////////////////
 
\mbox{}\hrulefill\ 
 

  \subsubsection [GC\_SomethingFunction1] {GC\_SomethingFunction1}


  This function does something to the input variable and returns 
   the result in the value variable.
  \\
  \\{\bf INTERFACE:}
\begin{verbatim}   FUNCTION GC_SomethingFunction1( input ) RESULT( value )\end{verbatim}{\bf INPUT PARAMETERS:}
\begin{verbatim}     INTEGER(ESMF_KIND_I4), INTENT(IN) :: input   ! Input variable \end{verbatim}{\bf OUTPUT PARAMETERS:}
\begin{verbatim}     INTEGER(ESMF_KIND_I4), INTENT(IN) :: value   ! Output variable\end{verbatim}{\bf REVISION HISTORY:}
\begin{verbatim}    21 May 2008 - R. Yantosca - Initial Version\end{verbatim}{\bf REMARKS:}
\begin{verbatim}   Protex is great!\end{verbatim}

\markboth{Left}{Source File: template\_routine.F90,  Date: Tue May 13 06:41:06 CST 2014
}

 
%/////////////////////////////////////////////////////////////
 
\mbox{}\hrulefill\ 
 

  \subsubsection{GC\_Routine.F90 }


  This routine takes in an input variable, does something to it, 
    and then sends out an output variable.
  \\
  \\{\bf INTERFACE:}
\begin{verbatim} SUBROUTINE GC_Routine( input, output )\end{verbatim}{\bf USES:}
\begin{verbatim}   USE GC_SomethingMod\end{verbatim}{\bf INPUT PARAMETERS:}
\begin{verbatim}   REAL(ESMF_KIND_R8), INTENT(IN) :: input    ! input variable\end{verbatim}{\bf OUTPUT PARAMETERS:}
\begin{verbatim}   REAL(ESMF_KIND_R8), INTENT(IN) :: output   ! output variable\end{verbatim}{\bf BUGS:}
\begin{verbatim}   None known at this time\end{verbatim}{\bf SEE ALSO:}
\begin{verbatim}   GC_SomethingMod.F90\end{verbatim}{\bf SYSTEM ROUTINES:}
\begin{verbatim}   None\end{verbatim}{\bf FILES USED:}
\begin{verbatim}   GC_SomethingMod.F90\end{verbatim}{\bf REVISION HISTORY:}
\begin{verbatim}   21 May 2008 - R. Yantosca - Initial version\end{verbatim}{\bf REMARKS:}
\begin{verbatim}   Protex is great!
   \end{verbatim}

%...............................................................
\end{document}
